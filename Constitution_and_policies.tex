\documentclass{article}

\begin{document}

\section{Interpretation}

\subsection{Definitions}  
\subsubsection{} The   following  definitions  clarify   terminology  in   the   constitution,  unless  context suggests  otherwise.  



i)      Academic  Councils  –  ratified  student  groups  representing  academic  

units in accordance with 5.1.  



ii)     Annual  Operating  Budget  –  a  budget  detailing line  by  line  the  sums  

 spent or received to a given account within the GSA.  



iii)    Association – the Graduate Students’ Association may also be referred  

to as the Association.  



iv)      Chair Electoral Officer – see policy 2.1.1  



v)      Executive  –  the body which is comprised of the President, the Vice- 

Presidents and the Aboriginal Liaison  



vi)      GSA Council – the group of registered ratifies student representatives  

 from all schools or departments. Academic Councils within the CGSR  

that comprise the authoritative decision-making entity of the GSA.  



vii)    Representatives –  ratified student representatives from each academic  

unit belonging to an Academic Council.  



viii)    Social groups – ratified graduate student groups or clubs in accordance  

with policy 4.3.2.1.  



  



2.  The Association and Executive of the Association  



2.1. The Association  



2.1.1. The name  of the organization  shall be the University  of Saskatchewan  

 Graduate  Students’    Association,    (the    abbreviation    of    which    is   

 “GSA”,   hereinafter   also referred to as the  Association).  



2.1.2. The purpose  of the GSA  shall be to:  



i)      Ensure its members have access to quality services that support their  

 academic success,  



ii)     Advocate for the unique needs and concerns of its members,  



iii)    Build a cohesive community among its members.  



2.1.3. The   GSA   is   the   representative   of   the   graduate   students   of   the  

University of Saskatchewan to the administration, faculty, and external  

   

4  



 Graduate Students’ Association  



    Constitution as revised on April 21, 2011  


----------------------- Page 5-----------------------

community.  It  shall  attempt  to  provide  professional,  academic,  and  

social  activities,  promote  awareness  of  issues  relevant  to  graduate  

students, and provide services.  



i)      The GSA recognizes that every student has the right to equal treatment  

with     respect     to    student     activities     and     organizations,       without  

discrimination because of race, ancestry, place of origin, colour, ethnic  

origin, citizenship, religion, creed, sex, sexual orientation, age, marital  

status,  family  status,  disability  or  the  receipt  of  public  assistant.  The  

association       further     recognizes       that   some      students      have     been  

historically  and  systematically  disadvantaged  on  the  above  grounds,  

with  resultant  under-representation  in  institutions  of  post-secondary  

education.      The     GSA      shall    strive   to   be    free   of    all  forms     of  

discrimination in all endeavors.  



ii)     The GSA shall exert every reasonable effort to ensure the accessibility  

of its activities in accordance with the grounds outlined in subsection  

i).  



iii)    Pursuant to the approval of GSA Council, a right under subsection i)  

and/or subsection ii) is not infringed by the efforts of individuals or the  

formation  and  organization  of  groups  designed  to  assist  and  support  

those students who have been the target of systematic discrimination  

pursuant to the grounds enumerated in subsection i) and shoes aim to  

achieve  equal  opportunity,  or  that  is  likely  to  contribute  to  the  

elimination of the infringement of rights under subsection i).  



iv)     The   GSA   shall   strive   to   provide   a   stimulating   and   accessible  

educational       experience       at  the    University      of   Saskatchewan that  

promotes diversity of intellectual representation and perspective.  



v)      Association  policy  shall  be  consistent  and  reflective  of  the  items  

outlined in subsection i) to v) inclusive.  



vi)      Section 2.1.3 i) shall be subject to evolve with the Canadian Charter of  

Rights and Freedoms.  



2.1.4. The GSA shall have control over all legitimate student enterprises of a  

non-academic nature which fall within its own purview.  



2.1.5. The GSA shall be carried on without purpose of gain for its members  

and  any  profits  or  other  accretions  to  the  organization  shall  be  used  

solely to promote the purpose of the organization as outlined in 2.1.2.  



2.1.6. This  constitution  shall  take  precedence  over  all  other  councils  and  

student organizations to which GSA members belong, unless otherwise  

specified in this constitution.  



  

5  



Graduate Students’ Association  



   Constitution as revised on April 21, 2011  


----------------------- Page 6-----------------------

  



2.2.The Executive of the Association  



2.2.1. The Executive of the GSA shall consist of a President, Vice-President  

(Student Affairs), Vice-President (Finance), Vice-President (External),  

Vice-President  (Academic), Vice-President (Operations and  

Communications) and the Aboriginal Liaison. The executive shall be  

voting  members  of  both  the  Executive  Council  and  GSA  Council.  

Executive terms shall run from May 1 to April 30.  



2.2.2. The  Executive  shall  run  individually  and  shall  be  elected  by  the  

student  body  at  large.  The  campaign  and  election  shall  be  held  in  

accordance  with  the  Policy  2.  Elections  and  Referenda.  Individuals  

running for Executive member positions must be members during their  

term in office.  



2.2.3. In the absence or unavailability of any one member of the Executive  

the remaining Executive members may appoint a replacement, who is  

an association member, subject to the approval of the two-thirds (2/3)  

of the members of the GSA Council present and voting at a meeting  

where notice of a motion of approval has been duly given. Should the  

nominee be rejected, the GSA Council may make an appointment, or  

require   the   remaining   Executive   members   to   propose   alternative  

nominees.  



2.2.4.  Should the GSA Council decide that the exercising of its authority to  

nominate candidates is undesirable, it shall call an election to  be held  

in accordance with Policy 2. Elections and Referenda.  



2.2.5. In  the  absence  or  unavailability  of  more  than  one  member  of  the  

Executive  or  in  the  event  that  all  offices  become  vacant,  the  GSA  

Council shall choose replacements whoc shall assume the powers and  

duties of those Executive members. An election shall be held as soon  

as is feasible, in accordance with Policy 2. Elections and Referenda.  



2.2.6. Duties of the Executive are listed in Policy 1. Executive Duties.  



  



2.3.Removal of the Executive members  



2.3.1. The   President   and  Vice-Presidents   may   be   removed   from   office  

individually, however more than one may be removed at a time  



2.3.2.  Such removal may be enacted by:  



i)      A  non-confidence  vote  by  GSA  Council.  Such  a  vote  must  be  the  

result of a motion that has had at least two weeks written notice to all  



 

       6  



Graduate Students’ Association  



   Constitution as revised on April 21, 2011  


----------------------- Page 7-----------------------

GSA  Council  members.  To  be  resolved  the  non-confidence  motion  

must  be passed by a two-thirds (2/3) majority of the total membership  

of GSA Council; or  



ii)     A referendum to impeach. Such a referendum shall be received at any  

time by GSA Council and must be supported by the signatures of two  

percent (2%) of the members of the GSA. Such a referendum shall be  

conducted by the Chief Electoral Officer on a date selected by him/her,  

separate from any other election or referendum. At least a two week  

notice of the referendum must be given to GSA members via posters  

and  a  website  update  but  in  no  instance  shall  more  than  one  month  

pass between submission  of the  impeachment  papers  and  the  date  of  

the balloting.  



2.3.3. If a non-confidence vote or an impeachment referendum is resolved in  

the  affirmative,  the  GSA  Council  shall  proceed  under  Policy  2.  

Elections  and Referenda  to  elect  a  new  Executive  member(s)  so that  

not more than one month passes before the new President and/or Vice- 

Presidents take office. Throughout the interim period the duties of the  

Executive shall be carried out by such person or persons as appointed  

by the GSA Council.  



2.3.4. In the event of a resignation, the resignation of a vice president of the  

Executive shall be considered tendered when given to the President. In  

the event the President is resigning, resignation shall be given to GSA  

Council. Resignations including written or email are valid.  



3.  Aboriginal \& Indigenous Graduate Student Council  



3.1.Aboriginal \& Indigenous Graduate Student Council  



3.1.1. The GSA within its organizational capacity will support the Aboriginal  

\& Indigenous Graduate Student Council (AIGSC) in representing the  

unique nature of issues relevant to its membership.  



3.1.2. Notwithstanding 3.1.1 in an event where each organization’s mandate  

is conflicting on an issue the GSA must take priority in representing  

the student body at large.  



3.2.Aboriginal Student Liaison  



3.2.1. An Aboriginal Student Liaison will be appointed by the Aboriginal \&  

Indigenous Graduate Student Council which will follow the duties as outline  

in Policy 1. Executive Duties.   



3.2.2.  The  Aboriginal  Student  Liaison  will  be  supervised  by  the  AIGSC.  

Removal  or  changes  to  the  position’s  mandate  will  be  at  the  AIGSA’s  

discretion.  





      7  



       Graduate Students’ Association  



  Constitution as revised on April 21, 2011  


----------------------- Page 8-----------------------

4.  Membership and Fees  



4.1.Members  



4.1.1. Members of the association shall be regular members only.  



i)      Regular  members  shall  be  all  graduate  students  registered  in  the  

College of Graduate Studies and Research (CGSR).  



ii)     Post-doctoral fellows and graduate student members of the Saskatoon  

Theological Union may become regular members if they register with  

the College of Graduate  Studies  and  Research and  pay  the  full  GSA  

fees.  



4.1.2. Each member may belong to a member Academic Council as well as  

the GSA Council. He/she may enjoy the rights and privileges of both  

organizations.  



4.1.3.  Students  may  be  a  member  of  both  the  Aboriginal  \&  Indigenous  

Graduate   Student   Council   (AIGSC)   and   any   Academic   Council  

representing a school of the CGSR.  



  



4.2.Rights, Privileges and Obligations  



4.2.1. The rights and privileges of regular members shall be:  



i)      To vote on all GSA elections and referenda;  



ii)     To hold offices or positions of employment within the GSA;  



iii)    To attend meetings of the GSA subject to the rules and procedures of  

this constitution;  



iv)     To move or second motions at such meetings;  



v)      To speak for or against a motion;  



vi)     To   vote   at   GSA   Annual   General   Meetings   or   Special   General  

Meetings ;  



vii)    To gain admission to and/or actively participate in any GSA sponsored  

event and/or program subject to the restrictions of that particular event  

and/or program.  



4.2.2. Except as may be directed by GSA Council  with regards to the GSA  

Operations Budget, no  members is empowered to make purchases in  

the name of the GSA or in any other way financially obligate the GSA  

until permission has been granted by the Board.  



4.2.3. Members may tender a resignation in writing, which  shall be effective  

upon acceptance thereof by GSA Council. In the case of resignation, a  

 

      8  



       Graduate Students’ Association  



  Constitution as revised on April 21, 2011  


----------------------- Page 9-----------------------

member  shall  remain  liable  for  payment  of  any  assessment  or  other  

sum levied or which became payable by him/her to the GSA prior to  

acceptance of his/her resignation.  



4.3.Fees  



4.3.1. GSA members shall be charged an annual fee to fund operations of the  

organization. The University of Saskatchewan shall be empowered to  

collect from all GSA members. The GSA Council shall be responsible  

for ensuring that said fees  are  expended in  a  manner  consistent with  

the purpose of the organization.  



4.3.2. At a GSA Council meeting in December, GSA Council shall approve  

the  GSA  fee  for  the  next  fiscal  year.  A  fee  will  be  proposed  by  the  

GSA Executive and must pass a two-thirds (2/3) vote of GSA Council.  



4.3.3. The GSA fee may be increased by no more than 5% plus the previous  

year’s  Consumer  Price  Index  (CPI)  without  a  referendum  of  the  

student body as outlined in Section 13 of this Constitution.  



  



5.  GSA Council  



5.1.Membership  



5.1.1. Voting  members  of    GSA  Council  shall  be  the  GSA  Executive  and  

ratified Representatives from the Academic Councils. The number of  

Representatives allocated to each Academic Council is established in  

section 5.1.6.  



5.1.2. Any school or department within the College of Graduate Studies and  

Research is entitled to have representation on GSA Council.  



5.1.3. Each  school  or  department  within  the  CGSR  may  have  only  one  

Academic Council. Representatives from the member council shall be  

chosen     in   a   manner      determined      by    the   member       council.    No  

representative  shall  represent  both  a  school  of  the  CGSR  and  the  

Aboriginal Indigenous Graduate Student Council (AIGSC).  



5.1.4. The AIGSC will also be included as a member of GSA Council and  

will be granted representation as outlined by section 4.1.5.  



5.1.5. A college may have a coherent Academic Council under the conditions  

of  5.1.3, and where any department within that College may have a  

separate  Academic Council.  



5.1.6. The  number  of  Representatives  for  future  Councils  will  be  based  on  

the enrolment numbers of the previous year provided by the Academic  

Councils      at   the    earliest    possible     convenience.       Representative  



      9  



       Graduate Students’ Association  



  Constitution as revised on April 21, 2011  


----------------------- Page 10-----------------------

allocation for each Academic Council will be:  



i)      One   representative  from   each  Academic   Council   with  up   to  24  

graduate students,  



ii)     Two representative from each Academic Council with up to 25 to 49  

graduate students,  



iii)    Three  representatives  from  each  Academic  Council  with  50  to  99  

graduate students,  



iv)     Four  representatives  from  each  Academic  Council  with  100  to  149  

graduate students,  



v)      Five  representatives  from  each  Academic  Council  with  150  to  199  

graduate students,  



vi)     Six  representatives  from  each  Academic  Council  with  200  to  249  

graduate students,  



vii)    Seven representatives from each Academic Council with 250 or more  

graduate students.  



5.2.Terms of Reference of the GSA Council  



5.2.1. The GSA Council is the ultimate decision-making body for the policy  

and political affairs of  the  GSA. GSACouncil  has  full power,  within  

the  restrictions  of  this  Constitution,  to  create,  alter  and/or  terminate  

any policy statements, which are considered to be GSA Policy, and to  

deal  with  any  reports,  recommendations  and/or  conclusions  of  any  

groups, committees and organizations which fall within the purview of  

the GSA.  



5.2.2. Without  in  any  way  restricting  the  generality  of  the  foregoing,  GSA  

Council  shall  aim  to  fulfil  its  mandate  by  debating  and  dealing  with  

any  reports,  including  their  recommendations  and  conclusions;  with  

motions,  as  submitted  by  the  GSA  Executive,  GSA  Council  or  the  

student body at large; and with policy decisions regarding the official  

stance of the GSA on any issue within the purview of  GSA Council.  

GSA  Council  shall  also  be  empowered  to  direct  the  Executive  to  

represent  the  GSA  where  it  is  deemed  necessary,  and  to  carry  out  

various duties in the interest of the student body.  



5.2.3. In all debate and decision-making the GSA Council shall be guided, as  

far as possible, by the objects contained in this Constitution.  



5.2.4. GSA Council shall have the authority, except as otherwise specified, to  

appoint representatives of the GSA to any committees, commissions,  

boards etc  



 

     10  



       Graduate Students’ Association  



  Constitution as revised on April 21, 2011  


----------------------- Page 11-----------------------

5.3.Election and Removal of GSA Council Members  



5.3.1. With the exception of the GSA Executive, the members of the Council  

shall be elected by their respective organizations according to the rules  

and regulations for the  election of Council  members, as specified by  

the individual faculty Academic Councils.  



5.3.2. Each Academic Council member is expected to act in the best interest  

of their constituency.  



5.3.3. All   Academic   Council   members   are   subject   to   removal   under  

subsection 5.3.4  on the following grounds:  



i)      Absence from four (4) or more duly called meetings;  



ii)     Just cause;  



iii)    Theft, fraud or embezzlement of funds;  



iv)     Ineligibility to be a member.  



5.3.4. Members  of  the  GSA  Council  can  only  be  removed  by  the  faculty  

association  which  they  represent,  according  to  the  procedure  for  

impeachment   which   governs   that   school   or   department’s   student  

association.  The  Chairperson  of  GSA  Council,  however,  is  charged  

with formally notifying both GSA Council and the appropriate faculty  

association  of  a  voting  council  member's  breach  of  subsection  5.5.3.  

Furthermore,       GSA      Council   may   make   recommendations   to   the  

respective faculty association so that appropriate action is taken .  



5.3.5. Notwithstanding  that,  by  virtue  of  subsection  5.3.4,  GSA  Council  

members   may   only   be   removed   from   the   GSA   by   the   faculty  

association  which  they  represent,  upon  notifying  GSA  Council  of  a  

voting  member's  breach  of  subsection  5.3.3  the  Chairperson  shall  

move  a  motion,  which  has  had  proper  notice,  that  the  offending  

Council   member's   voting   rights   be   denied.   If   such   a   motion   is  

approved by a majority of GSA Council members present and voting  

then the offending GSA Council member shall be denied his/her vote  

from  there  forward.  A  GSA  Council  member  denied  his/her  vote  in  

accordance with this subsection shall retain all other rights associated  

with  GSA  Council  membership  but  shall  not  be  counted  in  quorum.  

The  GSA  Council  may  reinstate  at  any  time  in  the  future  a  GSA  

Council member's vote which has been denied.  



5.3.6.  Subsection 5.3.4 does  not  apply  to  the  Executive  of the  Association,  

whose removal from office is governed by Section 2.3.  



  



 

      11  



Graduate Students’ Association  



   Constitution as revised on April 21, 2011  


----------------------- Page 12-----------------------

  



5.4. GSA Council Meetings and Quorum  



5.4.1. The  reference  source  for  all  points  of  order  to  produce  shall  be  

Robert’s Rules of Order, revised in its most recent edition.  



5.4.2. GSA Council will meet on a monthly basis. Meetings will be held in  

accordance with the Rules of Order as outlined in Policy 5. Meetings.  



5.4.3. GSA  Council  meetings  will  be  chaired  by  a  Chairperson,  whose  

responsibilities will also include acting as the Chief Electoral Officer  

and Chief Returning Officer.  



5.4.4. The      Vice     President     Operations       and    Communications   shall   be  

responsible  for  keeping  minutes  of  all  regular  meetings  of  the  GSA  

Council   and   for   ensuring  the   duplication  and   distribution   of   all  

minutes,  papers  and  reports  or  other  documents  to  GSA  Council  

members as outlined in Policy 3. Communications.  



5.4.5. The  meetings  of  the  GSA   Council  shall  be  open  to  the  public;  

however,  there  shall  be  GSA  to  closed  sessions  if  so  decided  by  a  

majority of the members present. If a closed session is held, the reason  

therefore   shall   be   announced   immediately   following   the   closed  

session.  



5.4.6. A  fifty  percent  (50%)  majority  plus  one  of  voting  GSA  Council  

members, not including proxy members, shall form a quorum for the  

transaction of business. If quorum is not maintained, the meeting shall  

be adjourned and the time and names of members still present shall be  

recorded in the minutes.  Alternate members are not permitted to vote  

in  GSA  Council  as  stated  in  reference  source  for  all  points  of  order  

declared in 5.4.1.  



5.4.7. All motions of the  GSA  Council shall be decided by a majority vote,  

with each Representative and Executive  member entitled to one vote.  

In  the  case  of  an  equality  of  votes,  the  Chairperson  shall  cast  the  

deciding vote. All votes at such meetings shall be taken by ballot with  

the approval of the Chairperson, or the support of a majority of  GSA  

Council  members  present.  If  no  demand  is  made,  the  vote  shall  be  

taken in the usual way by a show of hands.  



5.4.8. Any  GSA  Council  meeting  shall  adjourn  at  a  point  no  later  two  (2)  

hours after the time when it was scheduled to begin, less a motion to  

extend by thirty (30) minutes is passed by a two/thirds (2/3) majority  

at each thirty (30) minute interval for a maximum of two (2) intervals.  



5.4.9. All   GSA   Council   Meetings   will   be   recorded   and   publicized   in  



 

      12  



Graduate Students’ Association  



   Constitution as revised on April 21, 2011  


----------------------- Page 13-----------------------

accordance with Policies 3. Communication and 5. Meetings.  



5.5. Notice of GSA Council Meetings  



5.5.1. The President or any Vice-President of the Association shall have the  

power to call, at any time, a special meeting of GSA Council. Further,  

such meetings shall be called by a student at large upon receipt of a  

petition to do so signed by ten (10) GSA Council Members. Notice of  

a special meeting of GSA Council shall be forty-eight (48) hours.  



5.5.2. Notwithstanding  subsection  5.4.5,  the  President  of  the  GSA,  in  the  

case of an emergency, may call a special meeting of GSA Council at  

any  time  without  being  subject  to  the  forty-eight  (48)  hour  notice  

requirement.  



6.  Academic Councils  



6.1.Funding  



6.1.1. In order to qualify for funding, Academic Council must:  



i)      Have an updated constitution on file with the GSA,  



ii)     Have an updated membership list,  



iii)    Have  a  GSA  council  bank  account  where  at  least  two  GSA  Council  

members must sign all transactions which debit the account,  



iv)     Submit a written request for funding to the VP Finance,  



v)      Have a ratified representative in GSA Council,  



vi)     Have a list of their executive members and contact information,  



vii)    Actively fulfill its constitutional duties to its respective members  



6.1.2. Academic Councils will be funded by GSA fees as outlined in Policy  

5. Finances  



6.1.3. All  ratified  Academic  Councils  are  eligible  to  request  for  additional  

funding  by submitting a written request for additional funding to the  

Vice-President (VP Finance).  



6.2. Alternates for Academic Council members are not eligible to  vote  

    in GSA Council meetings in agreement with  5.4.1.  



7.  Financials  



7.1.Budgeting  



7.1.1. The fiscal year of the organization will run from May 1 to April 30. At  

the  end  of  the  closing  year  an  annual  budget  will  be  prepared  to  

summarize the previous year’s financial activity.  



 

     13  



       Graduate Students’ Association  



  Constitution as revised on April 21, 2011  


----------------------- Page 14-----------------------

7.1.2. The annual operating budget, starting May 1, will be approved at the  

Annual General Meeting in April of the previous fiscal year. It will be  

created  with  recommendations  from  all  GSA  Executive  that  are  in  

office during the April Annual General Meeting. The annual operating  

budget  must  include  with  it  the  breakdown  of  expenditures  and  

revenues that will occur each month in the same line items appearing  

the annual operating budget.  



7.1.3. A  monthly  budget  will  be  presented  to  GSA  Council  each  month  

outlining  the  revenues  and  expenditures  accrued  during  the  previous  

month, this budget will be delivered at the GSA Council meeting.  



7.1.4. Monthly  budgets  should  include  the  actual  year-to-date,  budgeted  

year-to-date and budgeted year total for each line item.  



7.2. Accounting practices  



7.2.1. Under  no  circumstances  will  the  annual  operating  budget  ,  approved  

by GSA Council, be adjusted by the Executive  without a motion in a  

GSA  Council meeting supported by a two-thirds (2/3) vote.  



7.2.2. Any auditor hired to review the GSA financials must be approved by a  

two-thirds (2/3) vote of GSA course Council.  



7.2.3. In    all   circumstances       the    use   of   a   miscellaneousline   item    is  

discouraged.  Only  in  circumstances  where  expenditures  or  revenues  

cannot be traced shall funds be allocated to a miscellaneous account.  



7.2.4. All revenues or expenditures must be allocated to line items within the  

budget that best represent their nature.  



  



8.  Committees  



8.1.1.  Standing committees are established to consider continuing questions.  

Once   established   they   shall   serve   continuously   with   progressive  

changes  in  membership.  The  committee  shall  continue  to  exist  until  

otherwise decided by Course Council.  



8.1.2.  Special  committees  are  established  for  the  purpose  of  examining  

questions   when   no   appropriate   standing   committee   exists.   Their  

memberships   and   terms   of   reference   shall   be   determined   by   a  

resolution of Course Council. A special committee shall be considered  

disbanded   following   the   reception   of   its   final   report   by   Course  

Council.  



8.1.3. The  terms  of  reference  of  any  committee  are  set  by  Course  Council  

and  may  only  be  extended  or  reduced  by  Course  Council  through  a  



 

      14  



Graduate Students’ Association  



   Constitution as revised on April 21, 2011  


----------------------- Page 15-----------------------

motion with proper notice.  



8.1.4. The  opinions  and  expressed  views  and  policies  of  GSA  Committees  

are not necessarily those of Course Council.  



8.1.5. The      Chairperson      of    each    committee       shall   be    elected    by    the  

membership  of  the  Committee,  usually  from  the  membership  of  the  

committee unless otherwise directed.  



8.1.6. All committee  appointments  shall  be  made  and/or  ratified  by  Course  

Council.  



8.1.7. Committees  are  required  to  present  progress  reports  and  working  

papers,  verbally  or  in  a  written  statement,  to  Course  Council,  and  

should Course Council feel that the committee is neglecting its duties,  

it shall take whatever action is considered appropriate.  



8.1.8. A committee report, which shall be presented to Course Council by the  

Chairperson, shall reflect the majority opinion of the committee.  



8.1.9. Reports and their accompanying recommendations may be received by  

a   motion   of   Course   Council.   Any   report   not   received   shall   be  

considered not to exist.  



8.1.10.     Reports     may     be   approved      in   principle,     separate    from     their  

accompanying  recommendations,  by  a  motion  of  Course  Council.  A  

report   that   is   received   according   to   subsection   8.1.9   does   not  

necessarily  have  to  be  approved.  Reports  are  approved  in  principle  

while   their   recommendations   are  approved   separately   as   specific  

policy  resolutions.  Should  there  be  more  than  one  recommendation,  

they  shall  be  considered  for  approval  on  an  individual  basis,  and  

Course  Council  shall  only  be  bound  by  those  recommendations  that  

are  approved.  A  report  without  any  recommendations  requires  no  

further proceedings with reference to it, subsequent to its reception and  

approval in principle, except on the direction of Course Council.  



8.1.11.     Without in any way limiting the generality of the foregoing, seven   

(7) standing committees of the Association shall include:  



i)      Budget Planning Committee  



a.  GSA President  



b.  VP Finance  



c.  At least 2 Academic Councilors  



ii)     Bursary Selection Committee  



a.  GSA President  



 

      15  



       Graduate Students’ Association  



  Constitution as revised on April 21, 2011  


----------------------- Page 16-----------------------

b.  VP Finance  



c.  VP Academic  



d.  At least 2 Academic Councilors  



iii)    Elections and Referenda Committee  



a.  GSA President  



b.  VP Operations  



c.  CEO (Chair Electoral Officer)  



iv)     Constitution Review Committee  



a.  GSA President  



b.  VP External Affairs  



c.  At least 2 Academic Councilors  



v)      Code of Ethics and Discipline Committee  



a.  CEO (Chair Electoral Officer)  



b.  GSA President  



c.  At least 1 Academic Councilor  



d.  At  least  1  GSA  member  who  is  not  an  Executive  or  Academic  

    Councilor  



vi)     Travel Grant Committee  



a.  VP Finance  



b.  VP Academic  



c.  At least 2 Academic Council members  



vii)    Gala Committee  



a.  2 GSA Executives chosen by the GSA President  



b.  At least 2 Academic Council members  



9.  Association   Annual   General   Meetings   and      Association   Special  

    Meetings  



9.1.General  



9.1.1. There shall be two (2) types of meetings of the GSA members: Annual  

General Meetings and Special General Meetings.  



9.1.2. Annual General Meetings and Special General Meetings shall be held  

on the campus of the University in a location determined by the GSA  

Council.  Such  meetings  shall  be  chaired  by  the  Chairperson  of  the  



       

    16  



      Graduate Students’ Association  



 Constitution as revised on April 21, 2011  


----------------------- Page 17-----------------------

GSA Council.  



9.1.3. Annual General Meetings and Special General Meetings will be held  

in accordance with Policy 5. Meetings.  



9.1.4. Annual  General  Meetings  and  Special  General  Meetings  may,  from  

time to time, be adjourned to any future time or to a different place.  

Such business may be transacted at such future meeting as might have  

been transacted at the original meeting from which such adjournment  

took place. No notice shall be required for the motion for adjournment.  

The  motion  for  adjournment  may  be  made  notwithstanding  that  no  

quorum is present.  



9.1.5. All members of the GSA:  



i)      may be present at any Annual General Meetings and Special General  

Meetings;  



ii)     may speak to any motion under consideration;  



iii)    may move or second motions;  



iv)     may exercise their voting privileges.  



9.1.6. Each  member  of  the  GSA  shall  be  entitled  to  one  vote  at  Annual  

General Meetings and Special General Meetings.  



9.1.7. Motions at all Annual General Meetings and Special General Meetings  

shall  be  decided  by  a  majority  of  votes  present  in  person  unless  

otherwise  required  by  this  Constitution.  All  votes  at  such  meetings  

shall be taken by ballot if so demanded by any member present, but if  

no  such  demand  be  made,  the  vote  shall  be  taken  by  hands.  A  

declaration  by  the  Chair  that  a  resolution  has  been  carried  or  not  

carried, and an entry to that effect in the minutes of the meeting shall  

be  admissible  in  evidence  as  prima  facie  proof  of  the  fact  without  

proof of the number or proportion of the votes accorded in favour of or  

against such resolution. Should a ballot be demanded, it shall be taken  

in such a manner as the Chair shall direct. In the case of an equality of  

votes, the Chair shall be entitled to a second or deciding vote. In such  

cases, the Chair shall vote to maintain the status quo where possible.  



9.1.8. Notice of the time and place of Annual General Meetings and Special  

General Meetings shall be given by posting the location and time on  

the GSA website and posting flyers throughout campus.  



9.1.9. Quorum  at  Annual  General  Meetings  and  Special  General  Meetings  

shall be fifty (50) members of the GSA. Should there be no quorum at  

such  a  meeting,  business  may  proceed  as  usual;  all  decisions  made  





     17  



       Graduate Students’ Association  



  Constitution as revised on April 21, 2011  


----------------------- Page 18-----------------------

shall be binding, except as follows:  



i)      Should any members of the GSA disagree with any decision made at  

the  meeting  they  may  present  a  petition  containing  the  signatures  of  

one  hundred  (100)  members  of  the  GSA  asking  for  a  replacement  

Special General Meeting;  



ii)     The  petition  shall  only  be  received  within  two  (2)  weeks  after  the  

adjournment of the impugned meeting;  



iii)    The replacement Special General Meeting shall be held within one (1)  

week of the receipt of the petition;  



iv)     The quorum at such a replacement Special General Meeting shall be  

fifty (50) members of the GSA, and if present the replacement Special  

General  Meeting  may  review  any  decision  of  the  impugned  meeting  

and by a majority vote confirm or reject it;  



v)      If  no  quorum  is  present  the  replacement  Special  General  Meeting  

cannot  convene  and  the  decisions  of  the  impugned  meeting  shall  

become binding.  



9.1.10.     Any GSA member may place a motion on the agenda for Annual  

General Meetings and Special General Meetings provided that:  



i)      there is a mover and a seconder who are both GSA members; and  



ii)     the  motion  with  mover  and  seconder  is  submitted  in  writing  to  the  

President  on  or  before  the  seventh  (7th)  day  before  the  day  of  the  

meeting.  



9.1.11.     The agenda for all Annual General Meetings and Special General  

Meetings  shall  be  available  to  all  GSA  members  at  least  one  week  

before the day of the meeting.  



9.1.12.     Motions may be added to the agenda by  presenting the motion to  

the Chair at the beginning of the Annual General Meeting and Special  

General Meeting. The motion will be accepted if:  



i)      it  is  the  Chair's  opinion  that  there  were  reasonable  circumstances  

which prevented the motion  from  being  submitted before  the agenda  

deadline; or  



ii)     a  two-thirds  (2/3)  majority  of  the  members  present  at  the  meeting  

agree to consider the motion.  



9.1.13.     Unless otherwise specified, Annual General Meetings and Special  

General  Meetings may  deal  with  any  matters normally  dealt with  by  

GSA  Council. The decisions of Annual General Meetings and Special  

General Meetings shall be binding in GSA Council Policy.  



     18  



       Graduate Students’ Association  



  Constitution as revised on April 21, 2011  


----------------------- Page 19-----------------------

9.2.Annual General Meetings  



9.2.1. The  Annual  General  Meeting  shall  be  held  in  April  every  year.  The  

date of such meeting must be after the Executive elections.  



9.3.Special General Meetings  



9.3.1. The  President  or  Vice-President  of  the  Association  shall  have  the  

power to call, at any time, an Special General Meeting. Such meetings  

may also be called at the discretion of  GSA  Council or shall be called  

upon  receipt  by  the  President  of  a  petition  to  do  so  signed  by  one- 

hundred (100) members of the GSA.  



  



10.  Rights, Privileges and Obligations of Member Association  



10.1.1.     The  member  associations  shall  recognize  the  obligations,  powers  

and jurisdiction of the GSA as granted by this Constitution.  



10.1.2.     The GSA shall exercise its jurisdiction under subsection 2.1.6 over  

the  member  associations  only  in  extreme  circumstances.  If  an  action  

by the GSA in this regard is opposed by a member association, it shall  

only  be  achieved  and  implemented  by  a  two-thirds  (2/3)  vote  of  the  

total membership of GSA  Council.  



10.1.3.     While   member   associations   are   primarily   responsible   to   their  

respective  faculties,  they  also  recognize  their  responsibility  to  all  

students at the University and to the GSA as the student government at  

the University.  



10.1.4.     The GSA recognizes the specified rights, privileges and obligations  

which are beyond the jurisdiction of the GSA. These rights, privileges  

and obligations shall include, but shall not be limited to, the following:  



i)      Every member association has jurisdiction over its own governmental  

structure,  and  its  aims  and  purposes.  As  a  consequence,  member  

associations   shall   have   authority   to   amend   their   constitution,   to  

administer  and  regulate  their  own  elections,  and  shall  control  the  

selection and tenure of their representatives on GSA Council;  



ii)     Every member association has jurisdiction over its own policies;  



iii)    Every  member association has jurisdiction over its own finances and  

their  administration.  The  member  association  shall  have  full  control  

over their budget and any respective fees.  



iv)     Every member association has the authority:  



a.  To send representatives to GSA Council meeting  



 

      19  



Graduate Students’ Association  



   Constitution as revised on April 21, 2011  


----------------------- Page 20-----------------------

b.  To    ensure    the    GSA     Executive      is   held   accountable       to  the  

    Constitution.  



c.  To run social events subject to the guidelines in this Constitution;  



d.  To publish any publication it wishes;  



e.  To  administer,  regulate  and  control  its  own  physical  space  in  

    accordance  with  the  regulations  of  the  University  and,  where  

    applicable, the GSA;  



v)      Every member association shall have authority over and responsibility  

for   its   own    orientation   program   as   regards   policies,   programs,  

regulations and specific events.  



vi)     Every member association has the right to prior consultation regarding  

any GSA Policies or constitutional amendments which might affect the  

memberassociation,      before     such     policies     or    constitutional  

amendments are presented to GSA Council.  



11.  Transition  



11.1.1.     The Executive shall be retired at the close of their one year term in  

office. At that time the Executive-Elect shall assume the powers vested  

in the offices of the Executive.  



11.1.2.     In accordance  with Policy  3. Communication the  Executive, prior  

to  the  election  of  their  successors,  shall  ensure  that  the  Transition  

Manual  is  up  to  date.  Further,  during  the  Transition  period,  the  

Executive-Elect shall refer to said Transition Manual.  



12.  Amendments  



12.1.       Constitutional Amendments  



12.1.1.     Unless  otherwise  stated  in  this  Constitution,  any  part,  section,  

subsection or paragraph of this Constitution may only be amended at  

an  Annual  General  Meeting  or  Special  General  Meeting,  or  by  a  

referendum  of  the  membership  as  outlined  in  section  13  of  this  

Constitution.  



12.1.2.     Amendments  required  because  an  organization  referred  to  in  this  

Constitution has undergone a name change, become defunct, or have  

otherwise left the GSA, shall be considered housekeeping and may be  

entered as a matter of course by the Vice President External. The Vice  

President External shall notify GSA Council of any such amendments.  



12.1.3.     All amendments or changes to this Constitution shall be published  

in  an  update  version  of  the  constitution  on  the  GSA  website  in  

accordance  with  Policy  3.  Communication  two  weeks  after  been  



     20  



       Graduate Students’ Association  



  Constitution as revised on April 21, 2011  


----------------------- Page 21-----------------------

ratified.  



12.1.4.     Until  such  amendment  is  given  effect,  this  Constitution  shall  

remain  in  force  and  be  binding  upon  the  GSA  as  regards  any  party  

acting on the faith thereof.  



12.2.       Policy Manual Amendments  



12.2.1.     Policy manuals may be amended at any GSA Council meeting were  

due notice has been given. Amendments to policy manuals must pass a  

two-thirds (2/3) vote by GSA Council.  



13.  Canadian Federation of Students  



13.1.1.     The GSA shall budget for and send appropriately sized delegations  

to  all  CFS  national  meetings,  including  but  not  limited  to  the  May  

Semi-Annual  meeting,  the  November  Annual  meeting,  the  National  

Graduate  Caucus  Stand-Alone  and  the  National  Aboriginal  Caucus  

Stand-Alone meeting. This shall happen through a discussion leading  

to  a  consensus  at  the  GSA  executive  level  or  if  needed,  by  an  

executive   vote   in   the   regular   executive   meetings.   The   Council  

approval is required if more than four (4) delegates are to be sent to  

any of these meetings.  



13.1.2.     The GSA shall budget  for and send appropriately sized delegations  

to all CFS Saskatchewan provincial meetings.  



14.  Elections and Referenda  



14.1.1.     The  GSA  Chief  Returning  Officer/Chief  Electoral  Officer,  in  the  

final instance, shall be responsible for all GSA elections and referenda.  

The  Chairperson  of  the  GSA  Council  shall  be  the  Chief  Returning  

Officer and Chief Electoral Officer.  



14.1.2.     The Chief Returning Officer/Chief Electoral Officer, who shall be  

appointed  by  the  Vice  President  Operations  and  Communications,  

subject to ratification by GSA Council, shall be generally responsible  

for  the  administration  of  all  GSA  elections  and  referenda.  Without  

limiting  the  generality  of  the  foregoing,  the  duties  of  the  Chief  

Returning Officer/Chief Electoral Officer shall be:  



i)      To print posters;  



ii)     To ensure that there is notification of an election or referendum in on  

the GSA website one week in advance.  



14.1.3.     The Chief Returning Officer/Chief Electoral Officer shall announce  

the  results  of  an  election  or  referendum  to  the  candidates  involved  

immediately  after  they  become  known.  The  results  may  be  made  





     21  



       Graduate Students’ Association  



  Constitution as revised on April 21, 2011  


----------------------- Page 22-----------------------

public only after notification has been given to the candidates. Should  

the candidates not be immediately available, the results would be made  

public no less than two (2) hours after the results have been confirmed  

by the Chief Returning Officer or the Chief Electoral Officer.  



14.1.4.     The  Chief  Returning  Officer/Chief  Electoral  Officer  shall  vote  

twenty-four  (24)  hours  in  advance  and  place  their  ballot  in  a  sealed  

envelope  to  be  deposited  with  the  Vice  President  Operations  and  

Communications. This envelope will be opened only in the case of a  

tie  vote.  Should  there  be  more  than  two  candidates,  or  teams  of  

candidates,  the  Chief  Returning  Officer/Chief  Electoral  Officer  shall  

indicate their order of preference on the ballot form.  



14.1.5.     All  campaigns  for  elections  and  for  referenda  shall  be  run  in  

conjunction   with   Policy   2.   Elections   and   Referenda.   Complaints  

regarding  the  violation  of  these  regulations  shall  be  brought  to  the  

attention of the Chief Returning Officer/Chief Electoral Officer, and, if  

they feels it is necessary, GSA Council. Complaints must be in writing  

and must be brought forward no later than seventy-two (72) hours after  

the last poll closes.  



14.1.6.     A referendum on any issue within the purview of the GSA or the  

member association may be called at any time by the GSA  Council or  

by   any   member   of   the   GSA.   Questions   may   be   placed   on   the  

referendum directly by GSA Council, or by a written request from any  

member  of  the  GSA  to  the  Chief  Returning  Officer/Chief  Electoral  

Officer that is supported by signatures of at least two percent (2%) of  

current GSA members.  



14.1.7.     The placement of all questions in the referendum shall be subject to  

GSA  Council  approval.  Normally  GSA  Council  shall  approve  any  

question  that  has  garnered  sufficient  signatures  in  support;  however,  

GSA  Council shall retain the authority to reject any question where it  

finds  compelling  reason  to  believe  that  the  passage  of  the  question  

would constitute a violation of the mission or operating statement of  

the  GSA;  a  violation  of  law;  a  contravention  of  University  policy;  a  

violation  of  contractual,  financial  or  other  obligations  undertaken  by  

the  GSA;  or  would  otherwise  pose  a  significant  threat  to  the  best  

interests  of  the  GSA  or  be  injurious  to  the  welfare  of  the  graduate  

students  at the  University of  Saskatchewan. Any  such  rejection shall  

require a two-thirds (2/3) vote in support from all voting members of  

GSA Council.  



14.1.8.     The wording of all questions to be placed on any GSA referendum  

shall  be  subject  to  the  approval  of  the  GSA  Council.  GSA  Council  



     22  



       Graduate Students’ Association  



  Constitution as revised on April 21, 2011  


----------------------- Page 23-----------------------

  shall have the authority to remove or amend misleading statements or  

  extraneous   promotional   content;   correct   errors   of   fact;   and   edit  

  technical errors of spelling and grammar.  GSA  Council shall not have  

  the authority to fundamentally change the substance of a question.  



14.1.9.Referenda decisions shall bind GSA policy, but shall not bind the  

  policy  of  member  associations  unless  so  provided  in  the  individual  

  constitutions of the member councils.  
\end{document}