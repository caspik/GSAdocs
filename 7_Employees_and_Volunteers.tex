7.1. Harassment and Discrimination 
 
 7.1.1.1 The GSA seeks to prevent harassment and discrimination by 
 educating members of the student community as to what constitutes 
 such behavior. 
 
 
 
 7.1.1.2 The GSA will always make known that complaints from any 
 student, employee, volunteer or executive member of the GSA will be 
 investigated seriously, will be treated with respect and will remain 
 confidential when appropriate. The initiation, pursuit and support of a 
 complaint will not be an intimidating experience 
 
 
 
 7.1.1.3 The GSA recognizes that all of its members have the right to be 
 free from harassment and discrimination. The GSA is respectful of, 
 and will adhere to, the University policies and the law in their entirety. 
 
 
 
 7.1.1.4 Harassment and discrimination includes sexual harassment, 
 harassment based on gender, race, ethnicity, religion, creed and 
 sexual orientation. Such harassment and discrimination has the 
 purpose or effect of unreasonably interfering with an individual's or a 
 group's work or academic performance, or of creating an intimidating, 
 hostile or offensive working, living or academic environment. 
 Individuals or groups who are not the direct target of the conduct in 
 question may also suffer harassment and discrimination as the result 
 of being present when such conduct takes place. 
 
 
 
 7.1.1.5 The GSA encourages people who are feeling uncomfortable with a 
 situation to discuss the issue with someone, It is recognized that 
 discussing the issue does not mean that a complaint is being lodged. If 
 a volunteer or an employee wishes to discuss and/or lodge a complaint 
 of harassment or discrimination they may do so by raising the concern 
 with their direct supervisor, the human resources manager, (the Vice 
 President Operations and Administration), or the GSA President. If the concerns are not resolved through those venues, they are encouraged 
 to bring their concern forward to the University. Ultimately, the 
 complainant is free to go wherever or to whomever they feel the most 
 comfortable. 
 
 
 
 7.2. Hiring Practices 
 
 7.2.1. Applications 
 
 
 
 7.2.1.1 All positions under the GSA will be widely advertised. 
 Postings will be distributed through email, the Student Employment 
 and Career Center Job Posting website and on the GSA website in 
 accordance with Policy 3. Communications . 
 
 
 
 7.2.1.2 Applications will be accepted for a minimum of two weeks. 
 
 
 
 7.2.1.3 Applicants will be asked to submit a cover letter and resume. When 
 appropriate, a job application may be created for the position. 
 
 
 
 7.2.1.4 No applicant will be asked to disclose information about race, 
 gender identity, sexual orientation, economic status or marital status. 
 
 
 
 7.2.1.5 Applicants may be asked about their student status and expected 
 time to remain a student, for the purposes of scheduling and long 
 range planning only. 
 
 
 
 7.2.1.6 Job applications will be stored in confidential files for two years 
 in accordance with the Policy 3.7 Confidentiality . 
 
 7.2.2. Interviewing 
 
 7.2.2.1 Interviews will be coordinated by the Vice President 
 Operations and Communication or their designate. 
 
 
 
 7.2.2.2 There will be no less than two and no more than five interviewers. 
 
 
 
 7.2.2.3 Applicants will be advised of the interview time, location and 
 requirements no later than 24 hours before the interview. 
 
 
 
 7.2.2.4 Interviews will take place in a private room. 
 
 
 
 7.2.2.5 Each applicant will be asked the same questions. Questions 
 stemming from the applicant’s response will differ between applicants. 
 
 
 
 7.2.2.6 Interviewers ’ notes will be kept on file with other job search 
 documents for two years in accordance with Policy 3.7 Confidentiality . 
 
 
 
 7.2.2.7 Interviewees will be notified of the selection procedure and when 
 they can expect to be contacted. 
 
 
 
 7.2.2.8 All candidates that are interviewed will be informed weather or 
 not they have been awarded the position. 
 
 
 
 7.2.3. Hiring 
 
 7.2.3.1 The selection will not be based on race, gender identity, 
 sexual orientation, economic status or marital status. 

 7.2.3.2 It is not a requirement that the GSA only hire graduate students, 
 though they will be given preference. 
 
 
 
 7.2.3.3 New employees will be provided with a letter of offer outlining 
 the terms and conditions of the position including: 
 
 i) Description of duties. 
 
 ii) Work schedule. 
 
 iii) Pay rate. 
 
 iv) Supervisory structure 
 
 v) Start date. 
 
 vi) Holiday and sick leave provision. 
 
 vii) Reference to the Employment and Volunteer Policy and 
 their rights as an employee. 

 7.2.4. Equal Opportunities 
 
 7.2.4.1 It is the policy of the GSA to treat all employees and job 
 applicants fairly and equally regardless of their sex, sexual 
 orientation, marital status, race, colour, nationality, ethnic or national 
 origin, religion, age, disability or union membership status. 
 
 
 
 7.2.4.2 Furthermore the GSA will ensure that no requirement or 
 condition will be imposed without justification which could 
 disadvantage individuals purely on any of the above grounds. 
 
 
 
 7.2.4.3 The policy applies to recruitment and selection, terms and 
 conditions of employment including pay, promotion, training, 
 transfer and every other aspect of employment. 
 
 
 
 7.2.4.4 Any act of discrimination by the GSA or employees or any failure to 
 comply with the terms of the policy will result in disciplinary action. 
 
 
 
 7.2.5. Disciplinary Policy 
 
 7.2.5.1 The GSA requires good standards of conduct from its employees 
 along with satisfactory standards of work. The GSAs disciplinary 
 procedure applies to any misconduct or failure to meet the standards 
 of performance or attendance. 
 
 
 
 7.2.5.2 The purpose of the GSAs disciplinary procedure is to be corrective 
 rather than punitive and it should be recognized that the existence 
 of its disciplinary procedure is to help and encourage employees to 
 achieve and maintain acceptable standards of conduct, attendance 
 and job performance and to ensure consistent and fair treatment for 
 all employees. 
 
 7.3. Terms of Employment 
 
 7.3.1.1. The terms of employment for GSA Employees shall be determined 
 at a formal Executive Committee meeting and set out in writing in an 
 employee contract. 
 
 
 
 7.3.1.2. Changes in employment contract terms are valid only if they have 
 been passed and carried in the form of a motion at a formal Graduate 
 Students' Association Executive Committee meeting where written minutes 
 have been produced. 
 
 
 
 7.3.1.3. The Employee should be consulted prior to the meeting by the 
 entire Executive for the sole purpose of producing a statement from the 
 Employee which indicates if the proposed increase or decrease in terms of 
 employment are acceptable or unacceptable to the Employee, and that said 
 statement will be presented at the formal Executive meeting. Further, if 
 this signed statement is produced and presented at the meeting, attendance 
 of the Office Administrator at the formal Executive meeting is optional. 
 
 
 
 7.3.1.4. A minimum of two weeks written notice or pay in lieu of notice will 
 be given to the Office Administrator regarding any changes in the terms of 
 employment. 
 
 
 
 7.3.1.5. Performance appraisals, changes in employment contract 
 terms, and the preservation of a positive work environment for the 
 Employee are the collective responsibility of the Executive Committee. 

 7.4. Employee Reviews 
 
 7.4.1.1 Employees will undergo a review three months after their position 
 with the GSA begins and annually. 
 
 
 
 7.4.1.2 Annual reviews will be conducted in March or April. 
 
 
 
 7.4.1.3 The review will be conducted by the President and the Vice 
 President Operations and Administration. 
 
 
 
 7.4.1.4 Reviews will cover current and past accomplishments and challenges 
 and a vision of what will be done next. 
 
 
 
 7.4.1.5 The employee review will be an opportunity for the Employee 
 to discuss challenges with the Vice President Operations and 
 Administration and the President. 
 
 
 
 7.4.1.6 The review will be documented and stored in the Employee’s 
 confidential file. 
 
 7.5. Responsibility of Supervisory Personnel 
 
 
 
 7.5.1.1 Everyone under the GSA in leadership positions will strive 
 to create an environment free of harassment and discrimination for 
 those for whom they are responsible. Included within the ambit of that 
 responsibility is an awareness of what constitutes impermissible 
 harassment and discrimination, knowledge of the procedures that are 
 in place for dealing with allegations of harassment and discrimination, 
 and cooperation in the processing of complaints. 
 
 
 
 7.5.1.2 It also means that supervisors will not condone or turn a blind eye 
 to activities within their areas of responsibility which violate the 
 rights of students, faculty or staff, that they will ensure that all those 
 for whom they have responsibilities are aware that any form of 
 harassment and discrimination in all its manifestations is prohibited, 
 and that any complaints will be attended to immediately and 
 effectively. 
 
 
 
 7.5.1.3 Any member of the GSA may seek informal assistance or advice 
 from any GSA executive member in a leadership role. All such 
 consultations will be confidential in accordance with the Policy 3.7 
 Confidentiality . 
 
 
 
 7.5.1.4 All GSA members with supervisory, managerial or leadership 
 responsibilities have the responsibility to advise anyone whom they 
 believe have been harassed of their options and the assistance 
 available through the GSA, the University of Saskatchewan and the 
 Government of Saskatchewan, Campus Security or the police 
 department. 
 
 7.5.1.5 Such personnel may, without revealing the identity of the person 
 involved, also seek advice from the University of Saskatchewan as to 
 how to proceed in those instances where a person alleging to have 
 been subject to harassment or discrimination is unwilling to take the 
 matter to an advisor. 
 
 
 
 7.5.1.6 In the case of apparent group or systemic harassment or 
 discrimination, such personnel may themselves take a complaint to an 
 advisor on behalf of those allegedly harassed or discriminated against. 
 
 7.5.1.7. The President shall preside over the operations of direct supervisors 
 of GSA Staff. In the event of a concern with the direct supervisor, the 
 GSA President will be the next in the chain of command. 