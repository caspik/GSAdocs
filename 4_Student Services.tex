4.1 Bursaries 
 
 4.1.1.1 The Bursary Fund will provide assistance to those graduate students who are ineligible for other awards (e.g. NSERC, 
 SSHRC etc.), who demonstrate difficulties in acquiring funding, community involvement and good academic standing. 
 
 4.1.1.2 Bursary documents, including applications will be kept in confidential files, in accordance with Policy 3.7 Confidentiality. 
 
 
 
 4.1.2. Eligibility 
 
 4.1.2.1. Applicants must be registered as a graduate student at the University of Saskatchewan in the term of award. 
 
 4.1.2.2. Students are only eligible to receive bursaries once in a year (once in every three terms). 
 
 4.1.2.3. Current executive are not eligible for GSA Bursaries. 
 
 
 
 4.1.3. Bursary Fund Amount and Distribution 
 
 4.1.3.1. The bursary fund shall be no less than $3 000 for each term. This will come from the GSA’s operational budget, donations and fund raising events. 
 
 4.1.3.2. The bursary fund will be distributed evenly between four bursary recipients. 
 
 4.1.3.3. The bursary selection committee may chose to re-distribute the allocation of the bursary fund in special circumstances. 
 
 4.1.4. Application Procedure 
 
 4.1.4.1 Applications will require the submission of: 
 
 i) A one page cover letter written by the applicant, outlining their situation and why they are deserving of the bursary. 
 
 ii) A completed bursary application form (available at the GSA office and on the website). 

 iii) One letter of recommendation from a faculty member in their department. 
 
 4.1.4.2 Applications will be accepted in hard copy only. 
 
 4.1.4.3 Application deadlines are as follows: 
 
 i) Fall term: Mid October. 
 
 ii) Winter term: Mid February. 
 
 iii) Summer term: Mid June. 
 
 
 
 4.1.5. Bursary Selection Process 
 
 4.1.5.1 The Graduate Students' Association Bursaries Selection Committee will be chaired by the VP Operations and Administration and will include the Vice President Finance, and a minimum of two (2) Course Councilors. Bursary applicants may not sit on the committee. A new bursary selection committee will be convened for each term. There will be only one representative from any given department. 
 
 4.1.5.2 Although all graduate students are encouraged to apply, preference will be given to students in the following situations: 
 
 i) Students that are ineligible for large scholarships. 
 
 ii) Part time students. 
 
 iii) Senior students who have exhausted the funding for their program. 
 
 iv) Students in need of child care services. 
 
 v) Students who have demonstrated financial need, good community involvement and academic performance. 
 
 4.1.5.3 The Bursary Selection Committee shall use a Bursary Selection Rubric that considers academics, dependent children, whether they are international students or not, if they have a disability and their income to rank applicants. The top ranking applicants will be awarded equal amounts. 
 
 4.1.6. Awarding Bursaries 
 
 4.1.6.1 Bursary recipients will be notified within two weeks of the deadline. 
 
 4.1.6.2 The GSA may announce that bursaries have been awarded, but will not release names of recipients. 
 
 4.2 Health and Dental Insurance 
 
 4.2.1.1 The Vice President External Affairs will act as the liaison between the health and dental insurance provider and course council and will report on information pertaining to the insurance plan. 
 
 4.2.1.2 All GSA members will have access to health and dental insurance through the GSA under terms outlined by the insurance provider. 
 
 4.2.1.3 Course Council will decide on the health and dental insurance provider and fees by a two-thirds (2/3) majority vote. 
 
 4.2.1.4 There shall be a health and dental committee chaired by the VP External Affairs consisting of at least one additional Executive member and at least one Academic Councilor who shall discuss health and dental changes and make a recommendation to GSA Council. 
  
 4.3 Social Groups 
 
 4.3.1.1 The Vice President Student Affairs will be responsible for overseeing the ratification of Social Groups 
 
 4.3.1.2 The purpose of a Social Groups Policy is to provide for the social and academic needs of GSA members. With the provision for funding in the operating budget of the GSA support is given to groups who: 
 
 i) Have a minimum of 75% of its total membership that are GSA 
 members 
 
 ii) Have members from more than one Academic Council. 
 
 4.3.1.3 Prospective campus clubs must complete and submit an application for ratification along with a copy of their constitution no later than September 30. 
 
 4.3.1.4 Social Group status will be effective from September 1 to August 31. 
 
 4.3.1.5 Social Group must reapply for campus club status each academic year. 
 
 4.3.1.6 Social Group will be ratified by a two-thirds (2/3) majority vote of GSA Council. 
 
 4.3.1.7 The constitution of GSA Social Groups must be in accordance with Robert’s Rules of Order. 
 
 4.3.1.8 A Social Group must: 
 
 i) Exist for the betterment of its members. 
 
 ii) Use membership fees and money garnered for the objectives outlined in the constitution of that club. 
 
 iii) Elect its executive in a democratic fashion. 
 
 iv) Hold public meetings. 
 
 v) Make its governing documents public to its membership and the GSA. 
 
 vi) Not exist for the purpose of discrimination or harassment of any group. 
 
 vii) Not exist for the financial betterment of its members. 
 
 viii) Not restrict membership on a basis of sex, race, gender identity etc., unless approved by GSA Council. 
 
 4.3.1.9 A Social Group not operating in accordance with the Social Groups Policy may have its membership as a campus club revoked, subject to a two thirds (2/3) majority vote of GSA  Council.
 
 4.3.1.10 In the event the application for ratification is rejected, the Social Group may launch an appeal to GSA Course Council by notifying the Vice President Student Affairs in writing. 
 
 4.3.1.11 Ratified Social Groups may have the privilege of: 
 
 i) Access to the GSA Commons and its facilities in accordance with Policy 6. GSA Commons. 
 
 ii) Applying to the GSA for funding grants. 
 
 iii) Advertizing news and events through the GSA with the approval of the Vice President Operations and Administration in accordance with Policy 3.Communications. 
 
 4.3.1.12 Social Groups are eligible for funding according to Policy 8.4 Social Groups Funding. 
 
 4.3.1.13 The GSA will uphold the confidentiality of individuals in the Social Group according to the Policy 3.7 Confidentiality. 
  
 4.4 Student Advocacy 
 
 4.4.1.1 Students will be able to approach any GSA staff or executive member with a sensitive question without fear of judgment, harassment or discrimination. 
 
 4.4.1.2 With the exception of imminent danger to the individual or individuals in their surrounding or abuse, student inquiries will remain confidential at the request of the student in accordance with the Confidentiality Policy. 
 
 4.4.1.3 The GSA will not assume the role of a counselling service, social services or disciplinarian, but will work with students to overcome academic and non-academic issues that they are facing. 
 
 4.4.1.4 The GSA Executive and Staff will have knowledge of the resources available to students and to the GSA in its role as a student advocate and will refer students when necessary. 
 
 4.4.1.5 The Vice President Student Affairs will act as an advocate in non academic grievances. 
 
 4.4.1.6 The Vice President Academic will act as an advocate in academic grievances. 
 
 4.5 Orientation 
 
 4.5.1.1 The Vice President Student Affairs will coordinate with the University in its efforts to orient incoming graduate students. 
 
 4.5.1.2 The Aboriginal and Indigenous Student Liaison will contribute to the orientation of indigenous graduate students. 