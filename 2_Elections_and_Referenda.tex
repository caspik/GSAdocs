2.1 Executive Elections 
 
 2.1.1 Chief Electoral Officer (CEO) Statement of Responsibilities 
 
 2.1.1.1 The CEO’s overriding priority shall be to ensure that elections are well advertized to the student body and are fair and impartial. The CEO will do this by: 
 
 i) Collecting all nomination forms and ensuring candidates have fulfilled the requirements stated in the forms. 
 
 ii) Explaining the rules and procedures of the election to all nominees and providing a package that includes these details. 
 
 iii) Advertizing election details in accordance with Policy 3. 
 
 Communication. 
 
 2.1.1.2 The CEO shall monitor and resolve any infractions of the election policy. 
 
 2.1.1.3 The CEO shall not endorse any candidate and must remain publicly neutral. 
 
 2.1.1.4 The CEO, also the current GSA Council Chair, is appointed by the Vice President Operation and Communication. 
 
 2.1.1.5 The CEO is responsible for any and all matters relating to election campaigning expenses and disputes. 
 
 2.1.1.6 The CEO shall only make a ruling after a proper investigation has been conducted. A sanction will only be issued where the CEO has determined that a violation has occurred and has compelling evidence. The CEO shall interview any individual deemed relevant, and ensure that the offending party or parties have the opportunity to respond to any allegation made.
 
 2.1.1.7 When a situation arises that is not explicitly considered by this consistent with its intentions. 
 
 2.1.1.8 The CEO shall, during the nomination and campaigning period, be available to receive any dispute or complaint. 
 
 2.1.1.9 The CEO shall organize and Chair a Candidates Forum to be held during Campaign week 
 
 2.1.1.10 Should it be necessary due to unforeseen circumstances to alter, extend or cancel previously set election dates, the CEO shall be empowered to do so. 
 
 2.1.1.11 The CEO shall declare any potential personal conflict of interest to GSA Course Council prior to validation day. Where a conflict of interest has been determined to exist, the CEO shall be required to appropriately alter their responsibilities, take a leave of absence or resign. 
 
 2.1.1.12 The CEO will receive an honorarium in the value of the graduate student hourly rate for each hour of work committed to the election process. 
 
 2.1.2 Elections Timeline 
 
 2.1.2.1 The Executive of the GSA shall be elected before April 15. 
 
 2.1.2.2 Nominations will be open for two weeks for all positions. Should no nominations be received, the nomination period can be extended by one week for unfilled positions. After an extended period, if no nomination is received then the Chief Electoral Officer shall refer the vacant position(s) to the next General Meeting or Course Council Meeting, whichever comes first, to fill the position(s) as follows: 

 i) Call for nomination to the vacant position(s) during the next general meeting of GSA Council meeting and direction from the GSA Executive or GSA Council regarding the election. 
 
 ii) If only one candidate is nominated, the candidate will present a short speech promoting themselves. GSA Council may vote by show of hands after the candidate has left the room. 
 
 iii) If two or more candidates are nominated for a given position, each candidate will present a short speech promoting themselves. After such time, GSA Council will vote to fill the vacant position. Voting will be done by secret ballot and will be counted by the GSA Council Chair, who is also the Chief Electoral Officer, and they will immediately release the results. 
 
 2.1.2.3 After nomination period has ended, and all nomination forms have been confirmed by the CEO the campaign period will begin the next day. The Campaign period will run for one week (5 business days). 
 
 2.1.2.4 The CEO will hold a Candidates Forum that is open to all GSA members. 
 
 2.1.2.5 Voting will be held at the end of the campaign period. The incumbent Vice President Operations and Communication will facilitate an online poll. 
 
 2.1.2.6 Candidates shall not engage in campaigning during the nomination period. 
 
 2.1.3 Nominations 
 
 2.1.3.1 All current members of the GSA are eligible to run for positions of the GSA Executive. 
 
 2.1.3.2 Nomination Packages are to be made available at the GSA office and on the GSA Website for the duration of the nomination period. These packages must contain the following information: 
 
 i) The position(s) to be contested for. 
 
 ii) A Nomination Form. 
 
 iii) A copy of the GSA Constitution, Bylaws and Policies. 
 
 2.1.3.3 Nomination Forms must be signed by: 
 
 i) A nominator and a seconder 
 
 ii) The nominee 
 
 iii) 5 supporting GSA members 
 
 2.1.3.4 Nomination signatures may not be collected in campus pubs, cafeterias, GSA offices, and GSA Commons. 
 
 2.1.3.5 Candidates shall take a leave of absence during the campaign period from all extracurricular activities that, in the judgement of the CEO, convey unfair advantage, or establish or imply a conflict of interest. Notices of Leave shall be provided to the CEO. 
 
 2.1.3.6 A nominee may withdraw his or her candidacy at any time before the end of the nomination period. Should a candidate decide to withdraw his or her candidacy during the campaign period, a notice of withdrawal in written form shall be given to the CEO. 
 
 2.1.4 Election Campaigning 
 
 2.1.4.1 No campaigning shall take place outside of the campaigning period. 
 
 2.1.4.2 Campaigning is defined as any activity that, in and of itself, serves to publicize or promote an individual’s candidacy in a GSA election. 

 2.1.4.3 Campaigning includes and is limited to the use of booked rooms for public gatherings, public appearances, emails with more than 5 recipients, distribution of posters, and websites. 
 
 2.1.4.4 The GSA will provide the copying of a maximum of fifty (50) campaign posters. 
 
 2.1.4.5 All campaign materials and promotions are subject to the approval of the CEO. A sample of all campaign materials shall be submitted to the CEO to be kept on file for the duration of the campaign. All posters must be stamped by the CEO to indicate approval. 
 
 2.1.4.6 The contents and methods of campaigning shall be above reproach. Candidates shall not misrepresent the character or policies of other candidates, nor shall they interfere in any manner with the campaign materials of other candidates. Candidates shall not make statements that they know are untrue. All campaigning is subject to the approval of the CEO. 
 
 2.1.4.7 Campaign posters shall be no larger than 11 inches by 17 inches. 
 There shall be a limit of fifty (50) posters approved per 
 team/candidate. 
 
 2.1.4.8 Should a campaign poster be torn down, the CEO may use their 
 discretion to approve a new poster to replace it, so long as no 
 more than 50 posters are on display at one time. 
 
 2.1.4.9 Posters may not be affixed to painted areas, doors, or glass. 
 Candidates are responsible for ensuring that campaigning 
 conforms to individual building policies. 
 
 2.1.4.10 The CEO will advertize and chair a Candidates Forum that is 
 open to all GSA members. 
 
 2.1.4.11 Candidate profiles will be posted on the GSA website. 
 
 2.1.4.12 No campaigning shall take place in a scheduled class or lab, 
 seminar. 
 
 2.1.4.13 No form of off-campus campaigning shall be permitted. 
 
 Candidates shall not place campaign materials on trees or utility 
 
 poles either on or off campus. 
 
 2.1.4.14 Campaign/promotional materials may not be distributed to 
 
 mailboxes on or off campus, nor shall any form of door-to-door 
 
 campaigning be permitted. 
 
 2.1.4.15 No candidate may campaign publicly inside campus pubs or 
 cafeterias. Campaigning shall be permitted in line-ups but shall 
 not occur past the point where patrons are requested to provide 
 identification for entrance. Campaigning in line-ups shall also be 
 subject to any applicable residence or university regulations. 
 
 2.1.4.16 Campaign materials shall not be distributed in campus pubs, 
 cafeterias, or GSA Commons. 
 
 2.1.4.17 Email addresses may be collected by candidates subject to the 
 approval from the student to distribute campaign information. All 
 email content shall be consistent with previously approved 
 campaign material and shall be above reproach. 
 
 2.1.4.18 At the end of the campaign period, each candidate or team of 
 candidates will be required to remove all campaign materials 
 produced on their behalf for the purposes of the election by 8:00 
 pm on the day preceding the first day of voting. Websites and 
 other online content may remain available, but may not be altered 
 after the 8:00 pm deadline. 
 
 2.1.4.19 No polls or surveys of public opinion regarding GSA Elections 
 shall be published or broadcast on the day(s) of voting. 
 
 2.1.4.20 Individuals who hold GSA Executive positions and who are 
 serving as campaign managers, advisors or individuals otherwise 
 closely associated with a team’s campaign, shall declare this to 
 the president and the CEO. The CEO, in consultation with the 
 President, may require an individual to assume either altered responsibilities or take a leave of absence, if in their judgment 
 conveys an unfair advantage, or establishes or implies a conflict 
 of interest. All leaves of absence shall last for the duration of the 
 campaign period. 
 
 2.1.4.21 GSA Executive members and employees shall not use any 
 resources of the GSA to endorse or speak against any election 
 candidate publicly. This includes, but is not limited to: 
 
 i) Publishing opinions using a designated GSA title. 
 
 ii) Using a GSA email account to disseminate information. 
 
 iii) Using the GSA logo on any material. 
 
 
 2.1.4.22 GSA Council shall be a neutral forum. No promotional materials 
 shall be displayed or distributed to Course Council. 
 
 2.1.4.23 Candidates are responsible for the conduct of their campaign 
 organization and its members. Any violation of elections policy 
 by said members shall be regarded as a violation by the candidate 
 and their nomination may be revoked at the discretion of the 
 CEO. 
 
 2.1.5 Election Expenses 
 2.1.5.1. No candidate shall use personal funds to acquire campaign 
 materials of any sort. Any such materials used to promote a 
 candidate will be considered unauthorized and such an incident 
 shall be remedied at the discretion of the CEO. 
 2.1.5.2. The GSA will cover all costs for printing up to fifty (50) posters. 
 
 2.1.6 Polling Procedures 
 2.1.6.1 Polling will be open to all graduate students through the PAWS 
 website. 
 
 2.1.6.2 Candidates will be elected if they receive the most votes. 
 
 2.1.6.3 In the case where only one nomination is received for a position, a 
 yes/no vote will take place. Candidates will be elected if 50%+1 of the voters are in favor. 
 
 2.1.6.4 The Candidates’ names and positions will all be displayed in the 
 same font and size. 
 
 2.1.6.5 A link will be provided to the candidate profile page created on 
 the GSA website. 
 
 2.1.6.6 Eight percent (8%) of the total student graduate body must vote 
 for a general election results to be legitimate. In cases where this 
 does not occur a new vote must be held. 
 
 
 2.2 Referenda 
 
 2.2.1. GSA Council 
 
 2.2.1.1 GSA Council shall ratify the dates of GSA referenda at least a 
 month before a referendum is held. 
 
 2.2.1.2 Signature requirements for referenda questions shall be ratified by 
 GSA Council prior to the commencement of the nomination 
 period. 
 
 2.2.1.3 GSA Council as a whole will act as a neutral body with respect to 
 all referendum issues. 
 
 2.2.2. Chief Returning Officer (CRO) Statement of Responsibilities 
 
 2.2.2.1 The CRO’s overriding priority shall be to ensure that referenda 
 are well advertized to the student body and are fair and impartial. 
 The CRO will do this by: 
 
 1. Collecting referenda question. 
 2. Advertizing referenda details in accordance with Policy 3. 
 Communication. 
 
 2.2.2.2 The CRO, also the current Course Council Chair, is appointed by 
 the Vice President Operation and Administration. 
 
 2.2.2.3 The CRO is responsible for any and all matters relating to referenda campaigning, expenses and disputes. In exercising their 
 duties, the CRO may consult with any other individuals deemed 
 relevant. 
 
 2.2.2.4 The CRO shall only make a ruling after a proper investigation has 
 been conducted. A sanction will only be issued where the CRO 
 has determined that a violation has occurred and has compelling 
 evidence regarding the identities of the perpetrators. The CRO 
 shall interview any individual deemed relevant, and ensure that 
 the offending party or parties have the opportunity to respond to 
 any allegation made. 
 
 2.2.2.5 When a situation arises that is not explicitly considered by this 
 document, the CRO shall interpret policy in a manner consistent 
 with its intentions. 
  
 2.2.2.6 The CEO shall, during the nomination and campaign period, be 
 available to receive any dispute or complaint. 
 
 2.2.2.7 Should it be necessary, due to unforeseen circumstances, to alter, 
 extend, or cancel previously set referendum dates, the CRO shall 
 be empowered to do so. 
 2.2.2.8 The CRO will receive an honorarium in the value of the graduate 
 student hourly rate for each hour of work committed to the 
 referendum process. 
 
 2.2.3. Timetable for Referenda 
 
 2.2.3.1 All referendum questions shall be held within a stated time period 
 once per year. All attempts shall be made to hold the referenda in 
 October. GSA Council at their discretion may open more than one 
 referendum period within an academic year. It is strongly 
 discouraged to open more than one referendum period unless 
 under extenuating circumstances. A second referendum period 
 can be opened with a two-thirds (2/3) majority vote by GSA 
 Council. 

 2.2.3.2 Each referendum period will be promoted through the GSA 
 website one month prior to the opening for nominations. A 
 referendum period must be held each term and the time period for 
 the acceptance of nominations will be set by the incumbent Vice 
 President Operations and Communications. If no nominations are 
 received no referendum will be held. 
 
 2.2.3.3 The incumbent Vice President Operations and Administration 
 shall issue a proclamation one month before the nomination 
 period outlining the following: 
 
 i) The opportunity to nominate a question for referendum. 
 
 ii) The dates and times of the nomination period. 
 
 iii) Where nomination packages can be obtained and 
 deposited. 
 
 iv) The dates of the vote. 
 
 v) The Executive must advertise the referendum question, 
 dates and polling method via all possible means of 
 communication, minimum the GSA website, posters, 
 email and PAWS announcement. 
 
 vi) The Executive must advertise the referendum question, 
 dates and polling method via all possible means of 
 communication, minimum the GSA website, posters, 
 email and PAWS announcement. 
 
 2.2.4. Nominations 
 
 2.2.4.1 Any GSA member can bring forward nominations of a referenda 
 question but must be affiliated with a ratified Academic Council. 
 
 2.2.4.2 The nomination period is purely for organizational purposes and 
 for collecting nomination signatures. No campaigning (as defined in Policy 2.2.5. Campaigning) shall occur during the nomination 
 period. 
 
 2.2.4.3 Nomination Packages are to be made available at the GSA office 
 on the opening day of the nomination period onwards. These 
 packages must contain the following information: 
 
 i) “GSA Referenda”. 
 
 ii) The formula for writing referenda questions. 
 
 iii) A Nomination Form. 
 
 iv) A copy of the constitution pointing out relevant policies 
 and procedures. 
 
 v) A campaign expense form. 
 
 2.2.4.4 Nomination Forms must be signed by fifty (50) current GSA 
 members. 
 
 2.2.4.5 Nomination signatures may not be collected in campus pubs, 
 cafeterias, or at any GSA common space. 
 
 2.2.4.6 The CRO shall keep available for public viewing in the GSA 
 Commons a list of questions approved and verified that are to 
 appear on the referendum. Such information will also be made 
 available on the GSA website at the close of the nomination 
 period. 
 
 2.2.4.7 An organization may withdraw its referendum question at any 
 time before the end of the nomination period. A notice of 
 withdrawal shall be given in written form to the CEO. 
 
 2.2.5. Referenda Campaigning 
 
 2.2.5.1 Campaigning shall not commence until the beginning of the 
 campaign period. 
 
 2.2.5.2 Campaigning is defined as any activity that, in and of itself, serves to publicize or promote an individual’s or 
 organization’s position in a GSA referendum. 
 
 2.2.5.3 Campaigning includes, but is not limited to, use of booked rooms 
 for public gatherings, public appearances, issue of policy 
 statements, distribution of promotional materials or information, 
 paid advertising in campus media, etc. 
 
 Should an event, organized prior to the nomination period for 
 reasons unrelated to the referendum, be scheduled to take place 
 during the campaign period, the organization responsible for it 
 can submit a request to the CRO asking that the event be 
 considered a non-campaign event. If the CRO determines that the 
 event cannot be rescheduled, the request may be granted, so long 
 as no mention of the campaign is made at the event. 
 2.2.5.4. All campaign materials and promotions are subject to the approval 
 of the CRO. A sample of all campaign materials shall be submitted 
 to the CEO to be kept on file for the duration of the campaign. All 
 posters must be stamped by the CEO to indicate approval. 
 2.2.5.5. The contents and methods of campaigning shall be above reproach. 
 Organizations shall not misrepresent the character or policies of 
 other organizations, nor shall they interfere in any manner with the 
 campaign materials of other candidates. Candidates shall not make 
 statements that they know are untrue. All campaigning is subject to 
 the approval of the CRO. 
 2.2.5.6. Campaign posters shall be no larger than 11 inches by 17 inches. 
 There shall be a limit of fifty (50) posters approved per 
 team/candidate. 
 2.2.5.7. Should a campaign poster be torn down, the CEO may use his/her 
 discretion to approve a new poster to replace it, so long as no more 
 than fifty (50) posters are on display at one time. 
 2.2.5.8. Posters may not be affixed to painted areas, doors, or glass. 
 Candidates are responsible for ensuring that campaigning 
 conforms to individual building policy. 
 2.2.5.9. No campaigning shall take place in a scheduled lecture or lab. 
 2.2.5.10. No form of off-campus campaigning shall be permitted. 
 Candidates shall not place campaign materials on trees or utility 
 poles either on or off campus. 
 2.2.5.11. Campaign or promotional materials may not be distributed to 
 mailboxes on or off campus nor shall any form of door-to-door 
 campaigning be permitted. 
 2.2.5.12. No campaigning will take place in a student residence. 
 2.2.5.13. No candidate may campaign publicly inside campus pubs or 
 cafeterias. Campaigning shall be permitted in line-ups but shall not 
 occur past the point where patrons are requested to provide 
 identification for entrance. Campaigning in line-ups shall also be 
 subject to any applicable residence or university regulations. 
 2.2.5.14. Campaign materials shall not be distributed in campus pubs, 
 cafeterias, or GSA Commons. 
 
 2.2.5.15 Email addresses may be collected by candidates subject to the 
 approval from the student to distribute campaign information. All 
 email content shall be consistent with previously approved 
 campaign material and shall be above reproach. 
 
 2.2.5.16 Each candidate or team of candidates will be required to remove 
 all campaign materials produced on their behalf for the purposes 
 of the election by 8:00 pm on the day preceding the first day of 
 voting. Websites and other online content may remain available, 
 but may not be altered after the 8:00 pm deadline. 
 
 2.2.5.17 No polls or surveys of public opinions regarding GSA Elections 
 shall be published or broadcast on the day(s) of voting. 
 
 2.2.5.18 Individuals who hold GSA positions or any elected leadership 
 position in its member association, and who are serving as 
 campaign managers, advisors or individuals otherwise closely 
 associated with a team’s campaign, shall declare this to their 
 supervisor and the CRO. The CRO, in consultation with the 
 supervisor, may require an individual to assume either altered 
 responsibilities or take a leave of absence, if in his/her judgment it conveys an unfair advantage, or establishes or implies a conflict 
 of interest. All leaves of absence shall last for the duration of the 
 campaign period. 
 
 2.2.5.19 GSA Executive members and employees shall not use any 
 resources of the GSA to endorse or speak against any election 
 candidate publicly. This includes, but is not limited to: 
 
 i) Publishing opinions using a designated GSA title. 
 
 ii) Using a GSA email account to disseminate information. 
 
 iii) Using the GSA logo on any material. 
 
 2.2.5.20 Course Council shall be a neutral forum. No promotional 
 materials shall be displayed or distributed at Course Council. 
 
 2.2.5.21 A group or party wishing to run a “No” campaign must register 
 with the CRO. Only one party may campaign against any 
 question. In the event that more than one party provides notice of 
 their intent to run a “No” campaign against the same question, the 
 CRO shall request that these parties merge to run a single 
 campaign. If the parties are unable to reach agreement, the CRO 
 shall determine which party shall run the “No” campaign. The 
 CRO shall rule in favor of the first party to provide written notice 
 of intent, unless they determine that a subsequent applicant has a 
 significantly greater interest in the outcome of the question, or 
 significantly greater willingness and ability to affect the outcome. 
 The CRO’s ruling in this matter is not subject to appeal. “No” 
 campaigns shall be provided with the same resources made 
 available to the group placing the question. 
 
 2.2.5.23 No member of the GSA shall be eligible to run multiple “No” 
 campaigns at the same time. 
 
 2.2.5.24 Campaign organizations are responsible for the conduct of their members and volunteers. Any violation of referenda policy by 
 said members and volunteers shall be regarded as a violation by 
 the campaign organization. 
 
 2.2.6. Referenda Expenses 
 
 2.2.6.1 No organization or group shall use any internal or external funds 
 to acquire campaign materials of any sort. Any such materials 
 used to promote a referendum question will be considered 
 unauthorized and such an incident shall be remedied at the 
 discretion of the CEO. 
 
 2.2.6.2 The GSA will cover all costs for printing all posters. 
 
 2.2.7. Polling Procedures 
 
 2.2.7.1 Polling will be open to all graduate students through the PAWS 
 website. In special circumstance where polling stations must be 
 used section 2.2.5 and 2.2.6 will be enforced. 
 
 2.2.7.2 A 50% + 1 majority vote binds the GSA to the commitments of 
 the referendum. 
 2.2.7.3 The quorum required for referendum results to be binding will be 
 no less than 8% and recommended by the Executive and subject 
 to debate and approval by a two-thirds (2/3) majority vote of the 
 GSA Council. Approval must occur at the same time as the 
 approval of the referendum question in accordance with the 
 Constitution. 
 2.2.7.4 In the event that a quorum condition is not approved by a two- 
 thirds (2/3) majority of the GSA Council, quorum shall be held at 
 8% of the total graduate student body.