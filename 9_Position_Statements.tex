----------------------- Page 1-----------------------

  



  



                                                          



                             University of Saskatchewan   



                         Graduate Students’ Association  



                                                          



                                                                               

                                                          



                                          Policy Manual   



                                  9. Position Statements  



                                                          



                                                          



                                                          



Created April 13, 2010  



Last Updated: April 26, 2010  



                                    Graduate Students’ Association  

                                   



                                  Position Statements Policy Manual  



                                                      9-1  


----------------------- Page 2-----------------------

                                          Table of Contents  



 8.1     Position Statements .................................................................................................. 3  



 8.2     Tuition Position Statement ....................................................................................... 3  



  



                                        Graduate Students’ Association  



                                     Position Statements Policy Manual  



                                                            9-2  


----------------------- Page 3-----------------------

                             8.1      Position Statements  



7.1.1.1 Position statements herein are binding for the GSA.  



7.1.1.2 Position statements will be included in this policy with a two-thirds (2/3) majority  

         vote at Course Council.   



7.1.1.3 Any  GSA  member  may  bring  forth  a  position  statement  for  considerations  by  

         Course Council.  



                       8.2      Tuition Position Statement  



Accepted by Course Council on February 25, 2010  



The  Graduate  Students’  Association  (GSA)  at  the  University  of  Saskatchewan  cannot  

support the notion of any future tuition increases for graduate students at this institution.  

We recognize that the University of Saskatchewan has one of the lowest graduate tuition  

rates amongst Canadian universities, but we  also recognize that the universities of many  

other wealthy nations require no tuition at all from their citizens. Not only do the citizens  

of  these  countries  benefit  from  this  opportunity  to  partake  in  higher  education,  but  the  

countries in which they live benefit from a “better quality of life” due in part to the effect  

of having a better-educated population. Given this, and given the disagreement that exists  

on campus regarding spending priorities, the GSA believes that there are alternatives to  

tuition increases for balancing our institution’s budget.  



Saskatoon  has  seen  a  dramatic  cost-of-living  increase  in  recent  years  that  negatively  

impacts  the  ability  of  graduate  students  to  afford  education.  Increasing  costs  for  food,  

transportation, and accommodation in the city of Saskatoon are leading to an increased  

financial burden for all students. An increase in tuition will only compound this burden  

and  push  students  away.  An  increased  financial  burden  does  not  just  mean  less  pocket  

money. It means many students who may otherwise be able to commit themselves fully to  

their studies will have to take additional employment. This is not just an inconvenience  

for students; this reduces the quality of scholarship coming out of the U of S.   



While one could argue that our lower tuition rates may not serve to bolster our prestige,  

the fact is that other universities can justify higher tuition only because they already have  

higher  scholarly  prestige.  Until  we  generate  the  prestige  to  put  us  on  par  with  more  

expensive universities, our lower costs will continue to be one of our key attractions.  



If the U of S wishes to become a competitor amongst Canada’s top schools, it needs to  

foster  an  environment  conducive  to  higher  quality  scholarship.  Scholars  are  most  

productive and progressive when they are free to focus all of their attention on their work.  

This demands both that they have access to the resources and facilities they require and  



                                 Graduate Students’ Association  



                               Position Statements Policy Manual  



                                                 9-3  


----------------------- Page 4-----------------------

that they be as free as possible of extraneous constraints. Tuition is exactly the sort of  

barrier this university should be working to remove from the path of talented scholars. It  

is certainly not something to be increased.  



The GSA strongly believes that the Provincial Government needs to take a stronger role  

in supporting the financial needs of the University and should increase its contribution to  

the budgets of the University. The GSA also very firmly believes that increased tuition  

would reduce the diversity of voices among the scholars of this institution by placing a  

further  impediment  in  the  way  of  disadvantaged  populations  and  is  contrary  to  best  

interests of the university and the broader community.  



  



The GSA believes that the University should view graduate students as investments in the  

both the social and the economic future of our community and work to make sure that  

tuition:  



         i)        Be reduced annually with an end-goal of zero tuition.                       



         ii)       Be considered a barrier to education and not necessary.  



         iii)      Should be equitable and continue to be assessed without differential fees  

                   for international students.  



         iv)       Approached with leadership in mind rather than comparison.  



         v)        Increases  should  be  brought  to  the  student  body  before  approval  and  

                   implementation in the interests of transparency and legitimacy.  



                                    Graduate Students’ Association  



                                  Position Statements Policy Manual  



                                                      9-4  

